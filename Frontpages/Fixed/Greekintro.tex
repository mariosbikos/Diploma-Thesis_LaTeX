% Πρώτη σελίδα: Τίτλος και στοιχεία φοιτητή

\pagestyle{empty}

%\noindent{\large\bf ΠΑΝΕΠΙΣΤΗΜΙΟ ΠΑΤΡΩΝ}\\
%ΤΜΗΜΑ ΗΛΕΚΤΡΟΛΟΓΩΝ ΜΗΧΑΝΙΚΩΝ\\
%ΚΑΙ ΤΕΧΝΟΛΟΓΙΑΣ ΥΠΟΛΟΓΙΣΤΩΝ\\
%Τομέας Τηλεπικοινωνιών \& Τεχνολογίας Πληροφορίας\\
%Εργαστήριο Ασύρματης Τηλεπικοινωνίας


\begin{minipage}[c]{0.69\textwidth}
\noindent{\large\bf ΠΑΝΕΠΙΣΤΗΜΙΟ ΠΑΤΡΩΝ} \\
ΤΜΗΜΑ ΗΛΕΚΤΡΟΛΟΓΩΝ ΜΗΧΑΝΙΚΩΝ \\
ΚΑΙ ΤΕΧΝΟΛΟΓΙΑΣ ΥΠΟΛΟΓΙΣΤΩΝ \\
Τομέας Τηλεπικοινωνιών \& Τεχνολογίας Πληροφορίας\\
Εργαστήριο Ενσύρματης Τηλεπικοινωνίας
\end{minipage}
\begin{minipage}[c]{0.26\textwidth}
\includegraphics[scale=0.52]{Files/Figures/upatras_logo.jpg}
%\includegraphics[scale=0.3]{Files/Figures/uni_logo.jpg}
\end{minipage}



\vskip0.2cm
\noindent\rule{\textwidth}{0.2mm}
\vskip1cm




\begin{center}
\LARGE\textbf{Διπλωματική Εργασία}\\
\Large
του φοιτητή του Τμήματος Ηλεκτρολόγων Μηχανικών και\\
Τεχνολογίας Υπολογιστών της Πολυτεχνικής Σχολής του\\
Πανεπιστημίου Πατρών
\end{center}
\vskip0.5cm
\begin{center}
\LARGE
Μπίκου Μάριου-Φωτίου του Κων/νου\\
~~\\
Αριθμός Μητρώου: 7323
\end{center}

\vskip1cm
\begin{center}
\LARGE
\underline{Θέμα}
\vskip0.3cm
\textbf{Δυναμική Προσομοίωση Εικονικών Αντικειμένων \\ για Εφαρμογές Επαυξημένης Πραγματικότητας}
\\
\Large 
\textbf{Εφαρμογή σε Σκάκι Επαυξημένης Πραγματικότητας}

\end{center}

\vskip1cm
\begin{center}
\Large
\underline{Επιβλέπων}
\vskip0.2cm
Κωνσταντίνος Μουστάκας, Επίκουρος Καθηγητής
\vskip1cm
\textbf{Αριθμός Διπλωματικής Εργασίας:}
\end{center}
\vfill
\centerline{
\Large Πάτρα, Ιούνιος 2015
}
\clearpage
\null\clearpage % Για μπρος-πίσω εκτύπωση ώστε ο τίτλος να βγει σε ένα φύλλο


% Δεύτερη σελίδα: Πιστοποίηση και υπογραφές καθηγητών

\begin{center}
\LARGE
\textbf{ΠΙΣΤΟΠΟΙΗΣΗ}
\vskip0.5cm
Πιστοποιείται ότι η διπλωματική εργασία με θέμα
\vskip0.5cm
\textbf{Δυναμική Προσομοίωση Εικονικών Αντικειμένων \\ για Εφαρμογές Επαυξημένης Πραγματικότητας}
\\
\Large 
\textbf{Εφαρμογή σε Σκάκι Επαυξημένης Πραγματικότητας
}
\vskip2cm
\Large
Του φοιτητή του Τμήματος Ηλεκτρολόγων Μηχανικών και Τεχνολογίας Υπολογιστών
\vskip1cm
Μπίκου Μάριου-Φωτίου\\
\vskip0.5cm
Αριθμός Μητρώου: 7323
\vskip2cm
Παρουσιάστηκε δημόσια και εξετάστηκε στο Τμήμα Ηλεκτρολόγων Μηχανικών και Τεχνολογίας Υπολογιστών στις ...../....../......
\end{center}

\vfill

\begin{minipage}[!t]{3.0in}
\Large
\begin{center}
Ο Επιβλέπων
\vskip1cm
Κωνσταντίνος Μουστάκας,\\ Επίκουρος Καθηγητής
\end{center}
\end{minipage}
\hspace{0.27in}
\begin{minipage}[!t]{3.0in}
\Large
\begin{center}
Ο Διευθυντής Τομέα
\vskip1cm
Νικόλαος Φακωτάκης,\\ Καθηγητής
\end{center}
\end{minipage}
\clearpage
\clearpage
\null\clearpage % Για μπρος-πίσω εκτύπωση ώστε η πιστοποίηση να βγει σε ένα φύλλο


% Τρίτη σελίδα: Περίληψη
\begin{Large}
\noindent \textbf{Αριθμός Διπλωματικής Εργασίας:}
\vskip0.03cm
\vspace{-3mm}
\begin{center}
\LARGE\underline{Θέμα:}\\
\textbf{ \Large Δυναμική Προσομοίωση Εικονικών Αντικειμένων\\
\vspace{-2mm}για Εφαρμογές Επαυξημένης Πραγματικότητας}
\\
\large 
\textbf{Εφαρμογή σε Σκάκι Επαυξημένης Πραγματικότητας
}

\vskip0.2cm

\begin{tabular*}{1.00\textwidth}{@{\extracolsep{\fill} }  l  r  }
  \Large Φοιτητής: Μπίκος Μάριος-Φώτιος & \Large Επιβλέπων: Κων/νος Μουστάκας
\end{tabular*}

\vskip0.2cm
\vspace{-3mm}
\LARGE\textbf{Περίληψη}
\end{center}
\vskip0.06cm
\vspace{-3mm}

Για την παραγωγή ρεαλιστικών προσομοιώσεων μεικτής πραγματικότητας και για την ενίσχυση της αίσθησης εμβύθισης του χρήστη σε αυτές, οι σύγχρονες προσεγγίσεις πρέπει, όχι μόνο να απεικονίζουν ρεαλιστικά τα εικονικά αντικείμενα, αλλά και να επιτρέπουν την φυσική αλληλεπίδραση με αυτά. Αντικείμενο της παρούσας διπλωματικής εργασίας είναι η αξιοποίηση και ανάπτυξη προηγμένων τεχνικών αντιστοίχισης του εικονικού περιβάλλοντος με το πραγματικό για την υπέρθεση και το χειρισμό εικονικών αντικειμένων σε προσομοιώσεις μεικτής πραγματικότητας. Για το σκοπό αυτό, χρησιμοποιήθηκε ένας σύχρονος αισθητήρας χρώματος-βάθους της Intel και παράλληλα υλοποιήθηκαν αλγόριθμοι για την αναγνώριση της χειρονομίας τσιμπήματος αξιοποιώντας τη σχετική θέση του αντίχειρα και του δείκτη για την αλληλεπίδραση με το εικονικό περιεχόμενο της σκηνής.  Τέλος, αναπτύχθηκε μια εφαρμογή μεικτής πραγματικότητας, όπου ο χρήστης μπορεί, ενώ βρίσκεται μπροστά από ένα πραγματικό τραπέζι, να παίξει σκάκι εναντίον του υπολογιστή μετακινώντας τα εικονικά πιόνια μόνο με τα χέρια του.



\textbf{Λέξεις Κλειδιά:} Επαυξημένη Πραγματικότητα, Χειρισμός Εικονικών Αντικειμένων, Αναγνώριση Χειρονομιών, Αλληλεπίδραση Ανθρώπου-Υπολογιστή


\end{Large}

\clearpage
\null\clearpage %

% Τρίτη σελίδα: Περίληψη
\begin{Large}
\noindent \textbf{Diploma Thesis No:}
\vskip0.03cm
\vspace{-3mm}
\begin{center}
\LARGE\underline{Title:}\\\textbf{ \Large Dynamic Simulation of Virtual Objects\\ \vspace{-2mm}for Augmented Reality Applications}
\\
\large 
\textbf{Development of an Augmented Reality Chess
}

\vskip0.2cm
\begin{tabular*}{1.00\textwidth}{@{\extracolsep{\fill} }  l  r  }
  \Large Student: Bikos Marios-Fotios & \Large Advisor: K.Moustakas
\end{tabular*}


\vskip0.2cm
\vspace{-3mm}
\LARGE\textbf{Abstract}
\end{center}
\vskip0.06cm
\vspace{-3mm}

In order to produce realistic mixed reality simulations and enhance immersion in augmented reality systems, solutions must not only present a realistic visual rendering of the virtual objects, but also allow natural hand interactions. The main goal of this Thesis is to utilize and introduce advanced techniques for the superimposition and manipulation of virtual objects over the view of the real world for mixed reality simulations. In this work, a modern RGB-Depth sensor, provided by Intel, was used and a pinch gesture detection algorithm was implemented, employing user’s thumb and forefinger to interact with the virtual content. Ultimately, a Mixed Reality Chess was developed, focused on providing an immersive experience to users, so that they are able to manipulate virtual chess pieces in front of a real table and play against a chess engine.


\textbf{Keywords:} Augmented Reality, Virtual Object Manipulation, Pinch-Gesture Recognition, Human-Computer Interaction

\setstretch{1.3}  % It is better to have smaller font and larger line spacing than the other way round

% Define the page headers using the FancyHdr package and set up for one-sided printing
\fancyhead{}  % Clears all page headers and footers
\rhead{\thepage}  % Sets the right side header to show the page number
\lhead{}  % Clears the left side page header
