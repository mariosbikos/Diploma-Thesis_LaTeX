\chapter{Κβαντομηχανική}

Η κβαντομηχανική είναι ένα μαθηματικό υπόβαθρο - ένα σύνολο κανόνων - για τη κατασκευή φυσικών θεωριών. Τα αξιώματα της κβαντομηχανικής είναι ο συνδετικός κρίκος των μαθηματικών με την κάθε φυσική θεωρία και είναι εκείνα τα οποία καθορίζουν την φύση και τις ιδιότητες των κβαντικών συστημάτων.

\underline{1ο Αξίωμα:} Σε κάθε απομονωμένο φυσικό σύστημα αντιστοιχεί ένας μιγαδικός διανυσματικός χώρος με εσωτερικό γινόμενο (χώρος Hilbert), ο χώρος καταστάσεων του συστήματος (state space). Το σύστημα περιγράφεται πλήρως από το διάνυσμα κατάστασης, το οποίο είναι μοναδιαίου μέτρου στο χώρο κατάστασης.

Το πιο απλό κβαντομηχανικό σύστημα είναι το qubit, του οποίου ο χώρος κατάστασης είναι ο $\mathbb{C^2}$. Αν $\ket{0}$ και $\ket{1}$ τα διανύσματα βάσης του χώρου, τότε οποιαδήποτε τυχαία κατάσταση είναι της μορφής $\ket{\psi} = \alpha\ket{0} + \beta\ket{1} \alpha,\beta \in \mathbb{C}$ (ιδιότητα της υπέρθεσης. Πρέπει επίσης να έχει μοναδιαίο μέτρο, άρα $\braket{\psi}{\psi} = 1 \leftrightarrow \alpha^2 + \beta^2 = 1$

\underline{2ο Αξίωμα:} Η εξέλιξη ενός κλειστού συστήματος στο χρόνο περιγράφεται πλήρως από ένα μοναδιαίο μετασχηματισμό. Με άλλα λόγια, η κατάσταση $\ket{\psi}$ του συστήματος τη χρονική στιγμή $t_0$ είναι ανάλογη της κατάστασης $\ket{\psi'}$ τη χρονική στιγμή $t_1$ με συντελεστή έναν μοναδιαίο τελεστή $U$ που εξαρτάται μόνο από τις χρονικές στιγμές $t_0,t_1$. \[\ket{\psi'} = U \ket{\psi}\]

Επομένως για να αλλάξουμε την κατάσταση ενός κβαντικού συστήματος στο χρόνο, τουλάχιστον σε θεωρητικό επίπεδο, ισοδυναμεί με την εφαρμογή των αντίστοιχων μετασχηματισμών στην αρχική κατάσταση, ώστε να καταλήξουμε στην επιθυμητή τελική κατάσταση. Για παράδειγμα , αν θέλουμε ένα qubit από την κατάσταση $\ket{0}$να το πάμε στην κατάσταση $\ket{1}$ αρκεί να πολλαπλασιάσουμε με τον τελεστή Pauli $\sigma_x$ : \[ \sigma_x\ket{0}\ = \ket{1} \Leftrightarrow 
\begin{bmatrix}
0&1\\
1&0\\
\end{bmatrix} \begin{bmatrix}1&0\end{bmatrix} = \begin{bmatrix}0&1\end{bmatrix}  \]

\underline{3ο Αξίωμα:} Οι κβαντικές μετρήσεις περιγράφονται πλήρως από ένα σύνολο τελεστών μέτρησης $\{M_m\}$ που δρούν στο χώρο κατάστασης του προς μέτρηση συστήματος. Αν το σύστημα είναι σε μια κατάσταση $\ket{\psi}$ πριν τη μέτρηση, τότε η πιθανότητα το αποτέλεσμα της μέτρησης να είναι $m$ είναι $p(m) = \matrixel{\psi}{\dagger{M_m}M_m}{\psi}$ και το σύστημα μετά τη μέτρηση θα βρεθεί στη κατάσταση \[\ket{\psi'} = \frac{M_m\ket{\psi}}{\sqrt{\matrixel{\psi}{M_m^{\dagger{}}M_m}{\psi}}}\].

Για παράδειγμα έστω qubit στην κατάσταση $\ket{\psi} = \alpha\ket{0} + \beta\ket{1}$. Τότε η πιθανότητα του αποτελέσματος της μέτρησης για τα δύο πιθανά αποτελέσματα καθορίζεται από τον τελεστή $M_0 = \ket{0}\bra{0}$ για τη μέτρηση $0$ και τον $M_1 = \ket{1}\bra{1}$ για την μέτρηση της κατάστασης $1$. Επομένως, π.χ. \[p(0) = \matrixel{\psi}{M_0^{\dagger{}}M_0}{\psi} = \braket{\psi}{0}\braket{0}{\psi} =  \begin{bmatrix}\alpha^{*}&\beta^{*}\end{bmatrix} \begin{bmatrix}1\\0\end{bmatrix} \begin{bmatrix}1&0\end{bmatrix} \begin{bmatrix}\alpha\\\beta\end{bmatrix} = \alpha^{*}\alpha = \alpha^2\] 
πράγμα που έρχεται σε συμφωνία με τον ορισμό του qubit στο 3ο Κεφάλαιο. Μετά τη μέτρηση η κατάσταση του συστήματος θα γίνει $\ket{\psi'} = \frac{\alpha}{\|\alpha\|}$, δηλαδή κάθε επόμενη μέτρηση θα δίνει την κατάσταση $\ket{0}$ με πιθανότητα $1$. (κατάρρευση κατάστασης λόγω μέτρησης)
