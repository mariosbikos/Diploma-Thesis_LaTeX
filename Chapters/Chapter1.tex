%*******10********20********30********40********50********60********70********80

% For all chapters, use the newdefined chap{} instead of chapter{}
% This will make the text at the top-left of the page be the same as the chapter

\chap{Εισαγωγή} \label{c:intro}


\section{Γενική Περιγραφή}


Ένας νέος κλάδος έρευνας, που στόχο έχει την απρόσκοπτη παρουσίαση πληροφοριών υπολογιστή στον πραγματικό κόσμο, όπως τον αντιλαμβάνεται ο άνθρωπος, είναι η Επαυξημένη Πραγματικότητα (Augmented Reality).

Οι εφαρμογές επαυξημένης πραγματικότητας μπορούν να μετατρέχουν τις κάμερες σε "μαγικούς φακούς", μέσα από τους οποίους πραγματικά και εικονικά αντικείμενα συνυπάρχουν στον ίδιο χώρο.
Αυτή η ψευδαίσθηση επιτυγχάνεται με την υπέρθεση εικονικού περιεχομένου πάνω από τον πραγματικό κόσμο. Για να βελτιωθεί η αίσθηση της παρουσίας σε περιβάλλοντα μεικτής πραγματικότητας, ένα σύστημα πρέπει να παρέχει ένα συγκεκριμένο επίπεδο εμβύθισης στο χρήστη. Αυτό μπορεί να επιτευχθεί μέσω της φυσικής αλληλεπίδρασης με τα εικονικά αντικείμενα, μία επιστημονική περιοχή, η οποία αποτελεί πρόκληση για περαιτέρω έρευνα.

Κατά καιρούς, μία μεγάλη ποικιλία τεχνικών αλληλεπίδρασης έχει χρησιμοποιηθεί για το χειρισμό των εικονικών αντικειμένως σε εφαρμογές επαυξημένης πραγματικότητας. Εντούτοις, οι περισσότερες προσεγγίσεις βασίζονται σε αλληλεπιδράσεις με δισδιάστατες οθόνες αφής και πατήματα πλήκτρων, με αποτέλεσμα να πάσχουν από μία περιορισμένη περιοχή εισόδου, αφού χρησιμοποιούν δισδιάστατους τρόπους εισόδου, για τρισδιάστατη αλληλεπίδραση.

Έτσι χάνεται η ψευδαίσθηση που χρειάζεται να έχουν οι χρήστες για την άμεση αλληλεπίδραση με τα εικονικά αντικείμενα στον πραγματικό κόσμο και δεν ωθούμαστε προς τη δυνατότητα απρόσκοπτων αλληλεπιδράσεων μεταξύ των χρηστών και του ψηφιακού περιεχομένου.

Πρόσφατες έρευνες έδειξαν ότι η αξιοποίηση φυσικών μεθόδων αλληλεπίδρασης ανάμεσα στα ελεύθερα χέρια χρηστών και τα εικονικά αντικείμενα, βελτιώνει την εμπειρία τους και αυξάνει τον ρεαλισμό του εικονικού περιεχομένου στον πραγματικό κόσμο. 
Σε σύγκριση, μάλιστα, με τις συμβατικές μεθόδους που επικεντρώνονται στη χρήση ενδιάμεσων συσκευών, οι διεπαφές που βασίζονται σε φυσικές χειρονομίες, παρέχουν μία πιο διαισθητική εμπειρία για εφαρμογές επαυξημένης πραγματικότητας. Παρά το γεγονός ότι έχουν παρουσιαστεί διαφορετικές προσεγγίσεις που χρησιμοποιούν τεχνικές όρασης υπολογιστών και επεξεργασίας εικόνας για την αναγνώριση χειρονομιών, τα τελευταία χρόνια, οι αισθητήρες βάθους και χρώματος όπως το Kinect ή το Leap Motion προσφέρουν εξαιρετικές δυνατότητες για την παρακολούθηση χειρονομιών και την ανίχνευση αλληλεπιδράσεων ελεύθερων χεριών που μπορούν να εφαρμοστούν σε περιβάλλοντα μεικτής πραγματικότητας, σε πολύ χαμηλό κόστος.

%already used-use refs
Most of the advances on augmented reality (AR) are related to tracking techniques and display technologies[13][4], however the interaction with virtual objects -usually limited to touch-screen displays- is still a challenging area that needs further improvements in order to address seamless interactions between the user and the augmented environment On previous works[6][7][8][9], the gesture recognition has been implemented using image processing techniques to detect hand gestures, using a single camera. The current possibilities that low-cost depth sensors like Kinect or Leap-Motion offer, can help to locate gestures within a certain space (our Augmented Reality space) and, in this manner, enable us to use the hand/fingers’ pose information to support the interaction with the virtual content. Research has been done in this area using Kinect[14][15], to infer, for example, the physical objects of a tabletop [16], and use the information to place the virtual content. The problem of interaction with virtual objects with physical gestures in an augmented reality environment is a challenging task that faces several problems that breaks the user experience [5]. According to [17], the hands-fingers interaction within augmented environments faces two major challenges: 1. The user’s fingers should be able to physically interact with virtual and real objects in an almost seamless way, a physically correct collision detection and processing with virtual objects is key[17]. 2. The mutual visual occlusion between virtual and real elements has to be of convincing quality. The correct occlusions between the user’s fingers and the virtual objects in the AR space should be as correct as possible[17].



%thomas-cie
By and large, the development of academic AR games for the indoor home market follows the concept of users playing games in one particular location, thus enabling the leveraging of current tracking technologies. Tracking technologies are more robust and accurate indoors, as the equipment does not suffer from as many environmental concerns; and many configurations tether people to their computers.

\section{Σχετικές Ερευνητικές Εργασίες}

%thomas-cie-apps of ar gaming
AR2Hockey was one of first AR games [Ohshima et al. 1998]. The game uses optical see-through HMD display technology for two players. The game is played on a standard table with landmarks. The landmarks allow for a hybrid optical tracking and the Polhemus’ Fastrack. The game basically supports the traditional form of air hockey, but replaces the physical pucks with virtual ones. As an extension to this,Mueller et al. developed an AR remote version of air hockey [Mueller et al. 2006]. Two remote physical air hockey tables, one for each player, provide the playing surface for the game. There is a video conferencing display across the middle of each table providing a real-time video feed of the other player. What makes this game different is that the users play with physical pucks. Once a puck is hit across the table, it is caught with a mechanism under the video conference display. The mechanism then automatically shoots the puck back in response to the shot from the other player. The game of pool (or billiards) has been investigated by a number of researchers as an application domain for AR. Jebara et al. developed the first mobile AR pool system [Jebara et al. 1997]. This is an HMDbased AR game, for which many of the first algorithms for image processing and physics engines for pool-based games were developed. This game served as a trainer for the end user by displaying AR information on the correct cue placement. Each of these games supports a physical gaming interface, adding to the evidence that AR incorporates both the physical and virtual worlds. AR2Hockey was a very early AR game, and as such required more expensive display and tracking equipment; but the price for both these forms of hardware has fallen dramatically. The Jebara et al. pool system also required a large structure, the pool table, to play the game upon. This system also incorporates a tutoring system for the players, including suggestions for shots.

A number of AR card games have been developed: Billinghurst et al. created an ARToolkit memory game, where the users flip physical cards [Billinghurst et al. 2000]. When a card is flipped over, a 3D graphic is displayed. The cards interact with each other by playing an animation when there is a match between the cards. This was the first AR game developed with the ARToolkit. Diaz et al. created a variant which employed hand gestures as the means of interaction [Diaz et al. 2006]. They used special cards to enable the system to sense card flipping by embedding Hall effect switches in the cards. BattleBoard is another example of a tabletop AR card game [Andersen et al. 2004]. This is an ARToolkit fiducial marker-based AR system which attaches virtual game pieces to the markers. One player employs an HMD with a camera and the second player views the game through a monitor. Battles are fought when pieces come in close proximity to each other, and thus activate AR animations. The Billinghurst et al. card game was designed for a public demonstration at an ACM SIGGraph conference with quick gameplay. BattleBoard employs a similar technology to the Billinghurst et al. card game, but BattleBoard’s design is more advanced, and is similar to a duelling card game. The Tankwar game was developed for more extended gameplay, investigating how AR could be employed for games with a more traditional time span.


%interaction in ar games
A drama-based game, Fac¸ade, was extended into an HMD AR version, AR Fac¸ade [Dow et al. 2007a, 2006, 2007b]. This game is a major break from traditional AR gaming ideas developed previously. This is a complex, real-life, role-playing game; very much like interactive theatre. Originally, the game was played on a traditional workstation; AR Fac¸ade is played on a HMD with a mobile backpack system, with gestures and voice as the main forms of interaction. The authors constructed virtual and physical representations of many of the game objects, such as walls and furniture, in an apartment. Objects that were manipulated by both the virtual characters and the physical players were presented as AR objects to the game player in the HMD. Due to the large area the game is played in a large area with an IS1200 tracking system. A Wizard of Oz methodwas employed to support user interactions to make formore robust gesture and speech processing systems during user studies. The authors found this form of interaction engaging for the user, but more research is required

%playstation app with board and monitor for AR
The Eye of Judgement game9 for the Sony PS3 is a third-person AR game which incorporates a digital video camera to capture the game board, and displays the AR version of the game on a television. This form of gaming requires the users to focus on both the physical game board for a part of the game and the AR version displayed on a monitor. Playing cards are placed on the game board to put them into play. Once on the game board, they act as fiducial markers to track augmented reality monsters and game pieces displayed on top of them. The games engine drives the 3D graphics of the cards once they are in play. This game was one of first commercial AR games on the market to make use of fiducial markers.


%Εργασίες με σκάκι ή pinch gestures για χειρισμό εικονικών αντικειμένων
Finally, Dorfm ¨ uller-Ulhaas and Schmalstieg developed an optical finger-tracking system with the goal of removing annoying wires and cables during the interaction. Their technology was demonstrated in the context of an augmented reality version of chess [Dorfmuller-Ulhaas and Schmalstieg 2001].


\section{Συνεισφορά}

Σκοπός της παρούσας διπλωματικής εργασίας είναι η μελέτη, υλοποίηση και ενσωμάτωση τεχνικών αναγνώρισης χειρονομιών στον τρισδιάστατο χώρο ως εναλλακτική είσοδος, για την ανάπτυξη μιας εφαρμογής μεικτής πραγματικότητας με βάση την ανίχνευση δεικτών και την αξιολόγηση των αποτελεσμάτων που μπορούν να επιτευχθούν. 

Στο πλαίσιο αυτό, ορίστηκαν τρεις στόχοι που πρέπει να έχουν εκπληρωθεί με το πέρας της εργασίας αυτής. Ο πρώτος στόχος αφορά στην κατανόηση της έννοιας της επαυξημένης πραγματικότητας και στον ορισμό μίας διαδικασίας που μπορεί να ακολουθηθεί προκειμένου να επιτευχθεί το τελικό αποτέλεσμα, με τεχνικές της όρασης υπολογιστών. Ο δεύτερος στόχος είναι η κατανόηση του θεωρητικού υποβάθρου των αλγορίθμων και των διαδικασιών που ορίστηκαν να εφαρμοστούν. Τέλος, ο τρίτος στόχος είναι ο προγραμματισμός των εφαρμογών με βάση τη διαδικασία που ορίστηκε και η αξιολόγηση των τελικών αποτελεσμάτων.


Προτείνουμε και υλοποιούμε μία μέθοδο, που επιτρέπει να προστεθεί μία τεχνολογία αναγνώρισης της χειρονομίας του τσιμπήματος(pinch) μέσα στο περιβάλλον της μεικτής πραγματικότητας, με στόχο τον απρόσκοπτο και εύκολο χειρισμό εικονικών αντικειμένων, χρησιμοποιώντας έναν αισθητήρα χρώματος-βάθους.
Τέλος επιδεικνύουμε τη λειτουργικότητα της μεθόδου ανίχνευσης χειρονομίας, δημιουργώντας μία εφαρμογή μεικτής πραγματικότητας. Η εφαρμογή αυτή είναι ένα σκάκι μεικτής πραγματικότητας, η οποία επιτρέπει στον χρήστη να μετακινεί εικονικά πιόνια με τα χέρια του.

Η κυριότερη συνεισφορά της διπλωματικής είναι η παρουσίαση μιας προσέγγισης που επιτρέπει την γενικότερη αξιοποίηση των δυνατοτήτων αισθητήρων χρώματος και βάθους σε περιβάλλοντα επαυξημένης πραγματικότητας, με απώτερο σκοπό την ευκολότερη αλληλεπίδραση με εικονικά αντικείμενα.
Στην προσέγγιση αυτή, χρησιμοποιήσαμε τον αισθητήρα τελευταίας τεχνολογίας RealSense 3D, που κυκλοφόρησε από την Intel το 2015, με αποτέλεσμα να μην υπάρχουν πολλές μελέτες σχετικά με τη χρήση του, κάτι που προσφέρει το πλεονέκτημα εκπόνησης εργασιών με βάση αυτή την τεχνολογίας.

Η χειρονομία τσιμπήματος είναι μία από τις πιο κοινές χειρονομίες για την αλληλεπίδραση με ψηφιακές διεπαφές. 
Μάλστα, Κατά τη διεξαγωγή ενός παιχνιδιού σκακιού, χρησιμοποιείται κατά κόρον από κάθε μέσο παίκτη για τη μετακίνηση των πιονιών. Επομένως για να δημιουργήσει κάποιος ένα σκάκι επαυξημένης πραγματικότητας, η χειρονομία τσιμπήματος είναι μία βασική χειρονομία που θα πρέπει να λάβει υπόψη του. 

Η αλληλεπίδραση με βάση τις χειρονομίες θα περιοριστεί στο σχεδιασμό και την υλοποίηση ενός απλού αλγορίθμου αναγνώρισης μίας χειρονομίας τσιμπήματος.

This is a clear example of employing an existing game design,
Snakes, onto a new platform, AR.
Overall, AR
is a growing method for presenting a gaming experience for users.



\section{Δομή της εργασίας}

Η εργασία διαιρείται σε έξι κεφάλαια, εκ των οποίων τα δύο πρώτα είναι κυρίως θεωρητικά, αλλά αναγκαία για την κατανόηση της έννοιας της επαυξημένης πραγματικότητας, της διαδικασίας που ακολουθήθηκε και του προγραμματιστικού υποβάθρου, ενώ το τελευταίο επικεντρώνεται στην ανάπτυξη των εφαρμογών, στην περιγραφή τους, στην παράθεση της μεθοδολογίας που ακολουθήθηκε και στην παρουσίαση των αποτελεσμάτων.
Συγκεκριμένα, το πρώτο κεφάλαιο αποτελεί μία εισαγωγή στην επαυξημένη πραγματικότητα.
Δίνεται ο ορισμός της και τα βασικά χαρακτηριστικά των εφαρμογών της, περιγράφεται η εξέλιξή της στο χρόνο και παρουσιάζονται οι κύριοι τομείς στους οποίους μπορεί να συνεισφέρει. Παράλληλα, αναφέρονται οι τεχνολογίες θέασης των επαυξημένων σκηνών αλλά και οι μέθοδοι τοποθέτησης των εικονικών πληροφοριών στον πραγματικό κόσμο. Τέλος, καταγράφονται οι κύριες προκλήσεις τις οποίες αντιμετωπίζουν οι δημιουργοί εφαρμογών επαυξημένης πραγματικότητας για την επίτευξη της ρεαλιστικής επαύξησης των σκηνών του πραγματικού κόσμου.
Το δεύτερο κεφάλαιο αναφέρεται στην ψηφιακή συνταύτιση χαρακτηριστικών σημείων, η οποία εφαρμόστηκε κατά την ανάπτυξη των εφαρμογών για την εύρεση των ομολογιών μεταξύ της εικόνας του επίπεδου πρότυπου αντικειμένου και της εικόνας του στο εκάστοτε στιγμιότυπο. Για λόγους πληρότητας, παρουσιάζονται στην αρχή του κεφαλαίου εν συντομία οι βασικές κατηγορίες της ψηφιακής συνταύτισης, ενώ το υπόλοιπο κεφάλαιο επικεντρώνεται στην ψηφιακή συνταύτιση χαρακτηριστικών σημείων. Η τελευταία διεξάγεται σε δύο γενικά στάδια, το πρώτο εκ των οποίων είναι η ανίχνευση και η περιγραφή των σημείων

Το τέταρτο κεφάλαιο αναφέρεται στο προγραμματιστικό περιβάλλον εργασίας. Ειδικότερα, γίνεται μία εισαγωγή στη γλώσσα προγραμματισμού των εφαρμογών C++, περιγράφονται τα βασικά στοιχεία ενός περιβάλλοντος ανάπτυξης προγραμμάτων C++, γίνεται αναφορά στο συγκεκριμένο περιβάλλον που χρησιμοποιήθηκε για την υλοποίηση των εφαρμογών και παρουσιάζονται οι βιβλιοθήκες που χρησιμοποιήθηκαν. Αυτές είναι η βιβλιοθήκη OpenCV οι σχετικές βιβλιοθήκες με τη διεπαφή προγραμματισμού εφαρμογών OpenGL, καθώς επίσης και η βιβλιοθήκη GLM: An Alias Wavefront OBJ file Library. Η βιβλιοθήκη OpenCV χρησιμοποιείται προκειμένου να υπολογιστούν τα στοιχεία του εξωτερικού προσανατολισμού του εκάστοτε στιγμιότυπου, αλλά και για τη βαθμονόμηση της μηχανής. Η ΟpenGL χρησιμοποιείται για τη σωστή τοποθέτηση του τρισδιάστατου μοντέλου στο στιγμιότυπο και τη θέαση της επαυξημένης σκηνής, ενώ η βιβλιοθήκη GLM χρησιμοποιείται για την εισαγωγή και εμφάνιση των τρισδιάστατων μοντέλων που διατίθενται σε μορφότυπο OBJ. Επιπλέον, στο κεφάλαιο αυτό επισημαίνονται κάποιες διαφορές των OpenCV και OpenGL, σχετικά με τα συστήματα συντεταγμένων που χρησιμοποιούν, και περιγράφονται οι μετασχηματισμοί που πραγματοποιούνται στην OpenGL προκειμένου ένα μοντέλο να παρουσιαστεί με το σωστό τρόπο σε ένα παράθυρο του υπολογιστή. Στο πέμπτο και τελευταίο κεφάλαιο της παρούσας εργασίας παρουσιάζονται οι εφαρμογές επαυξημένης πραγματικότητας που αναπτύχθηκαν. Περιγράφεται λεπτομερώς η μεθοδολογία που ακολουθήθηκε, αναφέρονται οι δυνατότητες των εφαρμογών και παρουσιάζονται τα αποτελέσματά τους. Επίσης, γίνεται αναφορά στις βαθμονομήσεις που διεξήχθησαν για την κάλυψη των απαιτήσεων των εφαρμογών και στο μορφότυπο OBJ, στο οποίο διατέθηκαν τα τρισδιάστατα μοντέλα επαύξησης της πραγματικότητας. Τέλος, μετά το κεφάλαιο αυτό ακολουθούν γενικά συμπεράσματα και σκέψεις για το μέλλον, που αποτελούν προτάσεις για μελλοντική επέκταση και βελτίωση των εφαρμογών.


Τα πρώτα Κεφάλαια \ref{c:complex} και \ref{c:finger} αποτελούν μια ανασκόπηση της θεωρίας των Κβαντικών Αποτυπωμάτων στο πλαίσιο της Πολυπλοκότητας της Επικοινωνίας. Το Κεφάλαιο \ref{c:complex} ορίζει βασικές έννοιες στην Κλασσική και Κβαντική Πολυπλοκότητα Επικοινωνίας, ενώ το Κεφάλαιο \ref{c:finger} πραγματεύεται ειδικότερα τα Κβαντικά Αποτυπώματα. Στο Κεφάλαιο \ref{c:crypto} διερευνούμε τα Κβαντικά Αποτυπώματα από τη σκοπιά της Κβαντικής Κρυπτογραφίας, εξηγούμε την ικανότητά τους στην απόκρυψη πληροφορίας και επιδεικνύουμε  τη χρήση τους ως Κβαντικές Ψηφιακές Υπογραφές. Προτείνουμε ένα σχήμα Κβαντικών Χρημάτων δημοσίου κλειδιού (public-key), το οποίο αποδεικνύεται ασφαλές έναντι κλασσικών και κβαντικών επιθέσεων. Τέλος, στο Κεφάλαιο \ref{c:exper} εξετάζουμε τις πειραματικές υλοποιήσεις των Κβαντικών Αποτυπωμάτων που έχουν προταθεί στην επιστημονική κοινότητα εστιάζοντας ιδιαίτερα σε μια πρόσφατη πειραματική διαδικασία που χρησιμοποιεί σύμφωνες καταστάσεις φωτός (coherent states) στην κατάσταση λειτουργίας time-bin (time-bin mode) \cite{constant}, η οποία καθιστά το πρωτόκολλο Κβαντικών Χρημάτων που προτείνουμε, εν μέρει πειραματικώς υλοποιήσιμο.

Στο πρώτο κεφάλαιο γίνεται μία αναφορά στο πώς οι μεθοδολογίες που αναπτύχθηκαν στα πλαίσια της διπλωματικής εργασίας μπορούν να χρησιμοποιηθούν σε πραγματικές εφαρμογές επαυξημένης πραγματικότητας.


Για λόγους ευκολίας κατά τη διάρκειας της ανάγνωσης, ο όρος επαυξημένη πραγματικότητα θα αντιστοιχεί τόσο σε μεικτές όσο και επαυξημένες πραγματικότητες.
