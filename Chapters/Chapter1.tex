%*******10********20********30********40********50********60********70********80

% For all chapters, use the newdefined chap{} instead of chapter{}
% This will make the text at the top-left of the page be the same as the chapter

\chap{Εισαγωγή} \label{c:intro}

Μία νέα επιστημονική περιοχή, που στόχο έχει την απρόσκοπτη παρουσίαση πληροφοριών και γραφικών, που παράγονται από υπολογιστές, στον πραγματικό κόσμο, έτσι όπως τον αντιλαμβάνεται ο άνθρωπος, είναι η Επαυξημένη Πραγματικότητα (Augmented Reality). 
Η τεχνολογία αυτή, επιτρέπει την επαύξηση του τρόπου με τον οποίο βλέπουμε, ακούμε και αντιλαμβανόμαστε το περιβάλλον γύρω μας, με νέους τρόπους που περιορίζονται μόνο από τη δύναμη της φαντασίας του ανθρώπινου νου και τις δυνατότητες των υπολογιστών. Οι εφαρμογές επαυξημένης πραγματικότητας μπορούν να μετατρέψουν, κατά κάποιο τρόπο, τις κάμερες σε "μαγικούς φακούς", μέσα από τους οποίους πραγματικά και εικονικά αντικείμενα συνυπάρχουν στον ίδιο χώρο. Αυτή η ψευδαίσθηση επιτυγχάνεται με την υπέρθεση του εικονικού περιεχομένου και του πραγματικού κόσμου. 


\section{Γενική Περιγραφή}



Για να βελτιωθεί η αίσθηση της παρουσίας σε περιβάλλοντα επαυξημένης πραγματικότητας, ένα σύστημα πρέπει να παρέχει ένα συγκεκριμένο επίπεδο "εμβύθισης" στο χρήστη. Τα εικονικά αντικείμενα οφείλουν να μοιάζουν ρεαλιστικά, ωστόσο αυτό μόνο δεν αρκεί. Η πραγματική "εμβύθιση" του χρήστη μπορεί να επιτευχθεί μέσω φυσικών διεπαφών που επιτρέπουν την αλληλεπίδραση με τα εικονικά αντικείμενα, μία επιστημονική περιοχή, η οποία αποτελεί πρόκληση για περαιτέρω έρευνα.


Κατά καιρούς, μία μεγάλη ποικιλία τεχνικών αλληλεπίδρασης έχει παρουσιαστεί για το χειρισμό των εικονικών αντικειμένων σε εφαρμογές επαυξημένης πραγματικότητας. Εντούτοις, οι περισσότερες προσεγγίσεις βασίζονται σε αλληλεπιδράσεις με δισδιάστατες οθόνες αφής και πατήματα πλήκτρων συσκευών, με αποτέλεσμα να περιορίζονται οι δυνατότητές τους, αφού χρησιμοποιούνται δισδιάστατοι τρόποι εισόδου, για τρισδιάστατες αλληλεπιδράσεις \cite{zhou2008trends} . Έτσι χάνεται η ψευδαίσθηση που χρειάζονται οι χρήστες για την άμεση αλληλεπίδραση με τα εικονικά αντικείμενα στον πραγματικό κόσμο και δεν ωθούμαστε προς τη δυνατότητα απρόσκοπτων αλληλεπιδράσεων με το ψηφιακό περιεχόμενο.


Ωστόσο, η αξιοποίηση φυσικών μεθόδων αλληλεπίδρασης ανάμεσα στα χέρια των χρηστών και τα εικονικά αντικείμενα, μπορεί να βελτιώσει την εμπειρία τους και να αυξήσει τον ρεαλισμό της σκηνής \cite{Kato2000}.  Σε σύγκριση, μάλιστα, με τις συμβατικές μεθόδους που επικεντρώνονται στη χρήση ενδιάμεσων συσκευών, οι διεπαφές που βασίζονται σε φυσικές χειρονομίες, παρέχουν μία πιο διαισθητική εμπειρία σε εφαρμογές επαυξημένης πραγματικότητας. Παρά το γεγονός ότι έχουν παρουσιαστεί διαφορετικές προσεγγίσεις που χρησιμοποιούν τεχνικές όρασης υπολογιστών και επεξεργασίας εικόνας για την αναγνώριση χειρονομιών, τα τελευταία χρόνια η αξιοποίηση αισθητήρων βάθους και χρώματος προσφέρει εξαιρετικές δυνατότητες για την παρακολούθηση χειρονομιών και την ανίχνευση αλληλεπιδράσεων χρησιμοποιώντας ως μέσο τα χέρια των χρηστών και η εφαρμογή τους σε περιβάλλοντα μεικτής πραγματικότητας μπορεί να επιτευχθεί με πολύ χαμηλό κόστος.


Οι κύριες προκλήσεις που εντοπίζονται κατά την ανάπτυξη εφαρμογών επαυξημένης πραγματικότητας, που έχουν ως στόχο την αλληλεπίδραση των εικονικών αντικειμένων με τα χέρια και τα δάκτυλα των χρηστών, αφορούν τον φυσικό χειρισμό εικονικών και πραγματικών αντικειμένων και την όσο το δυνατόν σωστότερη ποιότητα απεικόνισής τους, όταν τα εικονικά αντικείμενα αποκρύπτουν τα πραγματικά ή συμβαίνει το αντίθετο. 


Για λόγους ευκολίας κατά τη διάρκειας της ανάγνωσης, ο όρος επαυξημένη πραγματικότητα θα αντιστοιχεί τόσο σε μεικτές όσο και επαυξημένες πραγματικότητες.



\section{Συνεισφορά}


Σκοπός της παρούσας διπλωματικής εργασίας είναι η μελέτη, υλοποίηση και ενσωμάτωση τεχνικών ανίχνευσης χειρονομιών στον τρισδιάστατο χώρο, που μπορούν να αξιοποιηθούν ως εναλλακτική είσοδος για ποικίλες εφαρμογές, καθώς και η ανάπτυξη μιας εφαρμογής επαυξημένης πραγματικότητας και η αξιολόγηση των αποτελεσμάτων που μπορούν να επιτευχθούν. 



Προτείνεται και αναπτύσσεται μία μέθοδος, που επιτρέπει την αναγνώριση της χειρονομίας του "τσιμπήματος" (pinch-gesture), η οποία πραγματοποιείται από ένα χρήστη μέσα σε ένα περιβάλλον επαυξημένης πραγματικότητας, με στόχο τον απρόσκοπτο και εύκολο χειρισμό εικονικών αντικειμένων, χρησιμοποιώντας έναν αισθητήρα χρώματος και βάθους. Η μεθοδολογία που χρησιμοποιείται, επιτρέπει στους χρήστες να επανατοποθετούν εικονικά αντικείμενα στον τρισδιάστατο χώρο, γρήγορα και αποτελεσματικά, επιλέγοντάς τα και μετακινώντας τα μόνο με τα χέρια.


Τέλος επιδεικνύεται η λειτουργικότητα της μεθόδου ανίχνευσης χειρονομίας "τσιμπήματος", δημιουργώντας μία εφαρμογή μεικτής πραγματικότητας. Η εφαρμογή αυτή είναι ένα σκάκι επαυξημένης πραγματικότητας, όπου ο χρήστης μπορεί να μετακινεί εικονικά πιόνια με τα χέρια του και να παίζει το γνωστό επιτραπέζιο παιχνίδι με αντίπαλο τον υπολογιστή.


Η κυριότερη συνεισφορά της παρούσας διπλωματικής εργασίας είναι η παρουσίαση μιας προσέγγισης που επιτρέπει την γενικότερη αξιοποίηση των δυνατοτήτων αισθητήρων χρώματος και βάθους σε περιβάλλοντα επαυξημένης πραγματικότητας, με απώτερο σκοπό την ευκολότερη αλληλεπίδραση με τα εικονικά αντικείμενα. Στην συγκεκριμένη εργασία χρησιμοποιήσαμε τον αισθητήρα τελευταίας τεχνολογίας Intel\textregistered\ RealSense\texttrademark{} 3D F200, που κυκλοφόρησε από την Intel\textregistered\ το 2015, με αποτέλεσμα να μην υπάρχουν πολλές μελέτες σχετικά με τη χρήση του σε περιβάλλοντα επαυξημένης πραγματικότητας.


Η επαυξημένη πραγματικότητα είναι μία ταχέα αναπτυσσόμενη τεχνολογία που επιτρέπει τη δημιουργία μιας μοναδικής εμπειρίας βιντεοπαιχνιδιών για χρήστες όλων των ηλικιών. Η αλληλεπίδραση με βάση τις χειρονομίες θα περιοριστεί στο σχεδιασμό και την υλοποίηση ενός απλού αλγορίθμου αναγνώρισης μίας χειρονομίας τσιμπήματος. 


Η χειρονομία "τσιμπήματος" είναι μία από τις πιο γνωστές χειρονομίες για την αλληλεπίδραση με ψηφιακές διεπαφές. Κατά τη διεξαγωγή ενός πραγματικού παιχνιδιού σκακιού, χρησιμοποιείται κατά κόρον από κάθε παίκτη για τη μετακίνηση των πιονιών. Επομένως για να δημιουργήσει κάποιος ένα σκάκι επαυξημένης πραγματικότητας, η χειρονομία τσιμπήματος είναι μία βασική χειρονομία που θα πρέπει να λάβει υπόψη του. Η εργασία αυτή δείχνει, πώς ο σχεδιασμός ενός γνωστού επιτραπέζιου παιχνιδιού μπορεί να μεταφερθεί σε μία νέα πλατφόρμα, αυτή της επαυξημένης πραγματικότητας.







\section{Δομή της εργασίας}



Η εργασία διαιρείται σε πέντε κεφάλαια, εκ των οποίων τα δύο πρώτα είναι κυρίως θεωρητικά, αλλά αναγκαία για την κατανόηση βασικών εννοιών της επαυξημένης πραγματικότητας και της διαδικασίας που ακολουθήθηκε, ενώ το τρίτο επικεντρώνεται στην παράθεση της μεθοδολογίας των αλγορίθμων αναγνώρισης χειρονομιών που αναπτύχθηκαν, με στόχο το σχεδιασμό και την υλοποίηση μιας εφαρμογής σκακιού επαυξημένης πραγματικότητας. Στο τέταρτο κεφάλαιο παρουσίαζεται η αξιολόγηση των μετρήσεων και τα συμπεράσματα που προκύπτουν από τους αλγορίθμους και την εφαρμογή που αναπτύχθηκαν, ενώ στο τελευταίο κεφάλαιο γίνονται προτάσεις για μελλοντική επέκταση και βελτίωση των εφαρμογής σκακιού επαυξημένης πραγματικότητας.



Πιο συγκεκριμένα, το δεύτερο κεφάλαιο αποτελεί μία εισαγωγή στην επαυξημένη πραγματικότητα. Δίνεται ο ορισμός της και τα βασικά χαρακτηριστικά των εφαρμογών της, ενώ περιγράφεται η εξέλιξή της στο χρόνο. Παράλληλα, αναφέρονται οι τεχνολογίες επαύξησης αλλά και οι μέθοδοι τοποθέτησης των εικονικών πληροφοριών στον πραγματικό κόσμο. Επίσης παρουσιάζεται το μοντέλο της κάμερας που θα χρησιμοποιηθεί και οι βασικές μαθηματικές έννοιες της προβολικής γεωμετρίας, καθώς και η διαδικασία ανίχνευσης ειδικών δεικτών που ονομάζονται markers, ενώ το υπόλοιπο κεφάλαιο επικεντρώνεται στη χρήση και την αξιοποίηση μεθόδων αναγνώρισης χειρονομιών σε εφαρμογές επαυξημένης πραγματικότητας. Γίνεται επίσης αναφορά σε σχετικές ερευνητικές εργασίες που έχουν δημοσιευθεί.


Στο τρίτο κεφάλαιο περιγράφεται λεπτομερώς η μεθοδολογία και οι αλγόριθμοι που αναπτύχθηκαν με στόχο την υλοποίηση μιας εφαρμογής σκακιού επαυξημένης πραγματικότητας. Παράλληλα αναφέρονται τα εργαλεία ανάπτυξης της εφαρμογής και τα στοιχεία των συσκευών που αξιοποιήθηκαν για να γίνει πραγματικότητα. Επίσης, γίνεται αναφορά στη διαδικασία βαθμονόμησης που διεξήχθη για την κάλυψη των απαιτήσεων της εφαρμογής.


Το τέταρτο κεφάλαιο παρουσιάζει τα αποτελέσματα της διαδικασίας αξιολόγησης που πραγματοποιήθηκε μέσω μετρήσεων και ερωτηματολογίων σχετικά με την εφαρμογή. Με βάση τα αποτελέσματα, περιγράφονται τα συμπεράσματα και οι παρατηρήσεις σχετικά με τους περιορισμούς χρήσης της εφαρμογής και τα πλεονεκτήματα της.



Τέλος, μετά το κεφάλαιο αυτό ακολουθούν γενικά συμπεράσματα και σκέψεις για το μέλλον, που αποτελούν προτάσεις για μελλοντική επέκταση και βελτίωση των εφαρμογών.




