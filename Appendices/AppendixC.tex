\chapter{Κβαντική Εντροπία}

Ένας άλλος τρόπος περιγραφής των κβαντομηχανικών συστημάτων εκτός των διανυσμάτων κατάστασης είναι με τη χρήση τελεστών πυκνότητας ή αλλιως πίνακες πυκνότητας πιθανότητας. Η χρήση αυτόυ του τελεστή κρίνεται αναγκαία στην περίπτωση όπου δεν γνωρίζουμε από πριν την κατάσταση που βρίσκεται το σύστημα, αλλά ξέρουμε την πιθανότητα να βρίσκεται κάθε κατάσταση.

Έστω λοιπόν ενα κβαντικό σύστημα, το οποίο βρίσκεται σε μία από τις $Ν$ βασικές καταστάσεις του $\ket{\psi_i}$ με πιθανότητα $\rho_i$ . Καλούμε το σύνολο $\{\rho_i,\ket{\psi_i}\}$ μια συλλογή καθαρών καταστάσεων. Ο τελεστής πυκνότητας (πιθανότητας) που αντιστοιχεί τότε στο σύστημα ορίζεται ως: \[\rho \equiv \sum_i\rho_i\ket{\psi_i}\bra{\psi_i} \equiv
 \begin{bmatrix}\rho_1 & 0 & \cdots & 0 \\0 & \rho_2 & \cdots & 0 \\\vdots & \vdots & \ddots & \vdots \\0 & 0 & \cdots & \rho_N\end{bmatrix}
\]

Πλέον μπορούμε να ξαναγράψουμε και τα υπόλοιπα αξιώματα της Κβαντομηχανικής στη μορφή των τελεστών πυκνότητας.

Εξέλιξη στο χρόνο:   $ \rho' = \sum_i\rho_i\ket{\psi_i}\bra{\psi_i} \xrightarrow{U} \sum_i\rho_iU\ket{\psi_i}\bra{\psi_i}U^{\dagger{}} = U\rho U^{\dagger{}} $

Πιθανότητα Μέτρησης αποτελέσματος $m$: είναι ο μέσος όρος του τέλεστή μέτρησης $M_m$ πάνω στην κατανομή $\rho$, δηλαδή αν \[ p(m|i) = \matrixel{\psi_i}{M_m^{\dagger{}}M_m}{\psi_i} = Tr(M_m^{\dagger{}}M_m\ket{\psi_i}\bra{\psi_i}) \] \[p(m) = \sum_ip(m|i)\rho_i = \sum_i\rho_iTr(M_m^{\dagger{}}M_m\ket{\psi_i}\bra{\psi_i}) = Tr(\rho M_m^{\dagger{}}M_m)\]

Στην παραπάνω στοχαστική συλλογή $\rho$ ενός κβαντικού συστήματος, η εντροπία προκύπτει από την αβεβαιότητα που έχουμε για να βρίσκεται το σύστημα σε μια συγκεκριμένη κατάσταση. Στην κβαντομηχανική ορίζεται από τον τύπο του Von Neumann ως: \[S(\rho) = - Tr(\rho \log\rho) =  - \sum_{i=1}^{N}\rho_i \log \rho_i\] όπου $\rho$ ο δεδομένος πίνακας πυκνότητας πιθανότητας του συστήματος.
