%*******10********20********30********40********50********60********70********80

% For all chapters, use the newdefined chap{} instead of chapter{}
% This will make the text at the top-left of the page be the same as the chapter

\chap{Εισαγωγή} \label{c:intro}

Μία νέα επιστημονική περιοχή, που στόχο έχει την απρόσκοπτη παρουσίαση πληροφοριών και γραφικών, που παράγονται από υπολογιστές, στον πραγματικό κόσμο έτσι όπως τον αντιλαμβάνεται ο άνθρωπος, είναι η Επαυξημένη Πραγματικότητα (Augmented Reality). 
Η τεχνολογία αυτή, επιτρέπει την επαύξηση του τρόπου με τον οποίο βλέπουμε, ακούμε και αντιλαμβανόμαστε το περιβάλλον γύρω μας, με νέους τρόπους που περιορίζονται μόνο από τη δύναμη της φαντασίας του ανθρώπινου νου και τις τρέχουσες δυνατότητες των υπολογιστών. Οι εφαρμογές επαυξημένης πραγματικότητας μπορούν να μετατρέψουν, κατά κάποιο τρόπο, τις κάμερες σε "μαγικούς φακούς", μέσα από τους οποίους πραγματικά και εικονικά αντικείμενα συνυπάρχουν στον ίδιο χώρο. Αυτή η ψευδαίσθηση επιτυγχάνεται με την υπέρθεση του εικονικού περιεχομένου με τον πραγματικό κόσμο. 


\section{Γενική Περιγραφή}
%Πες λίγα λόγια για την AR(και πόσο σημαντική είναι), για το immersion που θελουμε να κανουμε achieve, για το interaction που προσδίδει realism, 


Για να βελτιωθεί η αίσθηση της παρουσίας σε περιβάλλοντα επαυξημένης πραγματικότητας, ένα σύστημα πρέπει να παρέχει ένα συγκεκριμένο επίπεδο εμβύθισης στο χρήστη. Τα εικονικά αντικείμενα οφείλουν να μοιάζουν ρεαλιστικά, ωστόσο αυτό μόνο δεν αρκεί. Η πραγματική εμβύθιση του χρήστη μπορεί να επιτευχθεί μέσω φυσικών διεπαφών που επιτρέπουν την αλληλεπίδραση με τα εικονικά αντικείμενα, μία επιστημονική περιοχή, η οποία αποτελεί πρόκληση για περαιτέρω έρευνα.

Κατά καιρούς, μία μεγάλη ποικιλία τεχνικών αλληλεπίδρασης έχει παρουσιαστεί για το χειρισμό των εικονικών αντικειμένων σε εφαρμογές επαυξημένης πραγματικότητας. Εντούτοις, οι περισσότερες προσεγγίσεις βασίζονται σε αλληλεπιδράσεις με δισδιάστατες οθόνες αφής και πατήματα πλήκτρων συσκευών, με αποτέλεσμα να περιορίζονται οι δυνατότητές τους, αφού χρησιμοποιούνται δισδιάστατοι τρόποι εισόδου, για τρισδιάστατη αλληλεπίδραση. Έτσι χάνεται η ψευδαίσθηση που χρειάζονται οι χρήστες για την άμεση αλληλεπίδραση με τα εικονικά αντικείμενα στον πραγματικό κόσμο και δεν ωθούμαστε προς τη δυνατότητα απρόσκοπτων αλληλεπιδράσεων με το ψηφιακό περιεχόμενο.


Ωστόσο, πρόσφατες έρευνες[βαλε link] έδειξαν ότι η αξιοποίηση φυσικών μεθόδων αλληλεπίδρασης ανάμεσα στα χέρια των χρηστών και τα εικονικά αντικείμενα, μπορεί να βελτιώσει την εμπειρία τους και να αυξήσει τον ρεαλισμό της σκηνής.  Σε σύγκριση, μάλιστα, με τις συμβατικές μεθόδους που επικεντρώνονται στη χρήση ενδιάμεσων συσκευών, οι διεπαφές που βασίζονται σε φυσικές χειρονομίες, παρέχουν μία πιο διαισθητική εμπειρία σε εφαρμογές επαυξημένης πραγματικότητας. Παρά το γεγονός ότι έχουν παρουσιαστεί διαφορετικές προσεγγίσεις που χρησιμοποιούν τεχνικές όρασης υπολογιστών και επεξεργασίας εικόνας για την αναγνώριση χειρονομιών, τα τελευταία χρόνια η αξιοποίηση αισθητήρων βάθους και χρώματος προσφέρει εξαιρετικές δυνατότητες για την παρακολούθηση χειρονομιών και την ανίχνευση αλληλεπιδράσεων χρησιμοποιώντας ως μέσο τα χέρια και η εφαρμογή τους σε περιβάλλοντα μεικτής πραγματικότητας μπορεί να επιτευχθεί με πολύ χαμηλό κόστος.

Οι κύριες προκλήσεις που εντοπίζονται κατά την ανάπτυξη εφαρμογών επαυξημένης πραγματικότητας που έχουν ως στόχο την αλληλεπίδραση των εικονικών αντικειμένων με τα χέρια και τα δάκτυλα των χρηστών, αφορούν τον φυσικό χειρισμό εικονικών και πραγματικών αντικειμένων και την όσο το δυνατό σωστή ποιότητα απεικόνισής τους, όταν τα εικονικά αντικείμενα αποκρύπτουν τα πραγματικά ή το αντίθετο.

%already used-use refs
Most of the advances on augmented reality (AR) are related to tracking techniques and display technologies[13][4], however the interaction with virtual objects -usually limited to touch-screen displays- is still a challenging area that needs further improvements in order to address seamless interactions between the user and the augmented environment On previous works[6][7][8][9], the gesture recognition has been implemented using image processing techniques to detect hand gestures, using a single camera. 

Research has been done in this area using Kinect[14][15], to infer, for example, the physical objects of a tabletop [16], and use the information to place the virtual content. 

The problem of interaction with virtual objects with physical gestures in an augmented reality environment is a challenging task that faces several problems that breaks the user experience [5]. According to [17], the hands-fingers interaction within augmented environments faces two major challenges: 1. The user’s fingers should be able to physically interact with virtual and real objects in an almost seamless way, a physically correct collision detection and processing with virtual objects is key[17]. 2. The mutual visual occlusion between virtual and real elements has to be of convincing quality. The correct occlusions between the user’s fingers and the virtual objects in the AR space should be as correct as possible[17].



Για λόγους ευκολίας κατά τη διάρκειας της ανάγνωσης, ο όρος επαυξημένη πραγματικότητα θα αντιστοιχεί τόσο σε μεικτές όσο και επαυξημένες πραγματικότητες.



\section{Συνεισφορά}
%ΑΝΑΦΕΡΩ ΤΙ ΑΚΡΙΒΩΣ ΕΚΑΝΑ ΣΤΗ ΔΙΠΛΩΜΑΤΙΚΗ(AR GAME APP, GESTURE DETECTION FOR VIRTUAL OBJECT MANIPULATION,ETC)

%wang-popovi
  
%--

Σκοπός της παρούσας διπλωματικής εργασίας είναι η μελέτη, υλοποίηση και ενσωμάτωση τεχνικών ανίχνευσης χειρονομιών στον τρισδιάστατο χώρο, που μπορούν να αξιοποιηθούν ως εναλλακτική είσοδος για ποικίλες εφαρμογές, καθώς και η ανάπτυξη μιας εφαρμογής επαυξημένης πραγματικότητας και η αξιολόγηση των αποτελεσμάτων που μπορούν να επιτευχθούν. 



Προτείνουμε και αναπτύσσουμε μία μέθοδο, που επιτρέπει την αναγνώριση της χειρονομίας του "τσιμπήματος" (pinch-gesture), η οποία πραγματοποιείται από ένα χρήστη μέσα σε περιβάλλον μεικτής πραγματικότητας, με στόχο τον απρόσκοπτο και εύκολο χειρισμό εικονικών αντικειμένων, χρησιμοποιώντας έναν αισθητήρα χρώματος και βάθους. Η μεθοδολογία που χρησιμοποιείται, επιτρέπει στους χρήστες να επανατοποθετούν εικονικά αντικείμενα στον τρισδιάστατο χώρο, γρήγορα και αποτελεσματικά, επιλέγοντάς τα και μετακινώντας τα μόνο με τα χέρια τους, χωρίς κάποιο άλλο βοήθημα.
Τέλος επιδεικνύουμε τη λειτουργικότητα της μεθόδου ανίχνευσης χειρονομίας "τσιμπήματος", δημιουργώντας μία εφαρμογή μεικτής πραγματικότητας. Η εφαρμογή αυτή είναι ένα σκάκι μεικτής πραγματικότητας, η οποία επιτρέπει στον χρήστη να μετακινεί εικονικά πιόνια με τα χέρια του και να παίζει το γνωστό παιχνίδι με αντίπαλο τον υπολογιστή.


Η κυριότερη συνεισφορά της διπλωματικής είναι η παρουσίαση μιας προσέγγισης που επιτρέπει την γενικότερη αξιοποίηση των δυνατοτήτων αισθητήρων χρώματος και βάθους σε περιβάλλοντα επαυξημένης πραγματικότητας, με απώτερο σκοπό την ευκολότερη αλληλεπίδραση με εικονικά αντικείμενα. Στην συγκεκριμένη εργασία χρησιμοποιήσαμε τον αισθητήρα τελευταίας τεχνολογίας RealSense\textregistered 3D F200, που κυκλοφόρησε από την Intel το 2015, με αποτέλεσμα να μην υπάρχουν πολλές μελέτες σχετικά με τη χρήση του.


Η επαυξημένη πραγματικότητα είναι μία ταχέα αναπτυσσόμενη τεχνολογία που επιτρέπει τη δημιουργία μιας μοναδικής εμπειρίας βιντεοπαιχνιδιών για χρήστες όλων των ηλικιών. Η αλληλεπίδραση με βάση τις χειρονομίες θα περιοριστεί στο σχεδιασμό και την υλοποίηση ενός απλού αλγορίθμου αναγνώρισης μίας χειρονομίας τσιμπήματος. Η χειρονομία "τσιμπήματος" είναι μία από τις πιο κοινές χειρονομίες για την αλληλεπίδραση με ψηφιακές διεπαφές. 


Κατά τη διεξαγωγή ενός πραγματικού παιχνιδιού σκακιού, χρησιμοποιείται κατά κόρον από κάθε παίκτη για τη μετακίνηση των πιονιών. Επομένως για να δημιουργήσει κάποιος ένα σκάκι επαυξημένης πραγματικότητας, η χειρονομία τσιμπήματος είναι μία βασική χειρονομία που θα πρέπει να λάβει υπόψη του. Η εργασία αυτή δείχνει πώς ο σχεδιασμός ενός γνωστού παιχνιδιού μπορεί να μεταφερθεί σε μία νέα πλατφόρμα, αυτή της επαυξημένης πραγματικότητας.







\section{Δομή της εργασίας}
%ΘΑ ΓΙΝΕΙ ΤΕΛΕΥΤΑΙΟ
Η εργασία διαιρείται σε έξι κεφάλαια, εκ των οποίων τα δύο πρώτα είναι κυρίως θεωρητικά, αλλά αναγκαία για την κατανόηση της έννοιας της επαυξημένης πραγματικότητας, της διαδικασίας που ακολουθήθηκε και του προγραμματιστικού υποβάθρου, ενώ το τελευταίο επικεντρώνεται στην ανάπτυξη των εφαρμογών, στην περιγραφή τους, στην παράθεση της μεθοδολογίας που ακολουθήθηκε και στην παρουσίαση των αποτελεσμάτων.
Συγκεκριμένα, το πρώτο κεφάλαιο αποτελεί μία εισαγωγή στην επαυξημένη πραγματικότητα.
Δίνεται ο ορισμός της και τα βασικά χαρακτηριστικά των εφαρμογών της, περιγράφεται η εξέλιξή της στο χρόνο και παρουσιάζονται οι κύριοι τομείς στους οποίους μπορεί να συνεισφέρει. Παράλληλα, αναφέρονται οι τεχνολογίες θέασης των επαυξημένων σκηνών αλλά και οι μέθοδοι τοποθέτησης των εικονικών πληροφοριών στον πραγματικό κόσμο. Τέλος, καταγράφονται οι κύριες προκλήσεις τις οποίες αντιμετωπίζουν οι δημιουργοί εφαρμογών επαυξημένης πραγματικότητας για την επίτευξη της ρεαλιστικής επαύξησης των σκηνών του πραγματικού κόσμου.
Το δεύτερο κεφάλαιο αναφέρεται στην ψηφιακή συνταύτιση χαρακτηριστικών σημείων, η οποία εφαρμόστηκε κατά την ανάπτυξη των εφαρμογών για την εύρεση των ομολογιών μεταξύ της εικόνας του επίπεδου πρότυπου αντικειμένου και της εικόνας του στο εκάστοτε στιγμιότυπο. Για λόγους πληρότητας, παρουσιάζονται στην αρχή του κεφαλαίου εν συντομία οι βασικές κατηγορίες της ψηφιακής συνταύτισης, ενώ το υπόλοιπο κεφάλαιο επικεντρώνεται στην ψηφιακή συνταύτιση χαρακτηριστικών σημείων. Η τελευταία διεξάγεται σε δύο γενικά στάδια, το πρώτο εκ των οποίων είναι η ανίχνευση και η περιγραφή των σημείων

Το τέταρτο κεφάλαιο αναφέρεται στο προγραμματιστικό περιβάλλον εργασίας. Ειδικότερα, γίνεται μία εισαγωγή στη γλώσσα προγραμματισμού των εφαρμογών C++, περιγράφονται τα βασικά στοιχεία ενός περιβάλλοντος ανάπτυξης προγραμμάτων C++, γίνεται αναφορά στο συγκεκριμένο περιβάλλον που χρησιμοποιήθηκε για την υλοποίηση των εφαρμογών και παρουσιάζονται οι βιβλιοθήκες που χρησιμοποιήθηκαν. Αυτές είναι η βιβλιοθήκη OpenCV οι σχετικές βιβλιοθήκες με τη διεπαφή προγραμματισμού εφαρμογών OpenGL, καθώς επίσης και η βιβλιοθήκη GLM: An Alias Wavefront OBJ file Library. Η βιβλιοθήκη OpenCV χρησιμοποιείται προκειμένου να υπολογιστούν τα στοιχεία του εξωτερικού προσανατολισμού του εκάστοτε στιγμιότυπου, αλλά και για τη βαθμονόμηση της μηχανής. Η ΟpenGL χρησιμοποιείται για τη σωστή τοποθέτηση του τρισδιάστατου μοντέλου στο στιγμιότυπο και τη θέαση της επαυξημένης σκηνής, ενώ η βιβλιοθήκη GLM χρησιμοποιείται για την εισαγωγή και εμφάνιση των τρισδιάστατων μοντέλων που διατίθενται σε μορφότυπο OBJ. Επιπλέον, στο κεφάλαιο αυτό επισημαίνονται κάποιες διαφορές των OpenCV και OpenGL, σχετικά με τα συστήματα συντεταγμένων που χρησιμοποιούν, και περιγράφονται οι μετασχηματισμοί που πραγματοποιούνται στην OpenGL προκειμένου ένα μοντέλο να παρουσιαστεί με το σωστό τρόπο σε ένα παράθυρο του υπολογιστή. Στο πέμπτο και τελευταίο κεφάλαιο της παρούσας εργασίας παρουσιάζονται οι εφαρμογές επαυξημένης πραγματικότητας που αναπτύχθηκαν. Περιγράφεται λεπτομερώς η μεθοδολογία που ακολουθήθηκε, αναφέρονται οι δυνατότητες των εφαρμογών και παρουσιάζονται τα αποτελέσματά τους. Επίσης, γίνεται αναφορά στις βαθμονομήσεις που διεξήχθησαν για την κάλυψη των απαιτήσεων των εφαρμογών και στο μορφότυπο OBJ, στο οποίο διατέθηκαν τα τρισδιάστατα μοντέλα επαύξησης της πραγματικότητας. Τέλος, μετά το κεφάλαιο αυτό ακολουθούν γενικά συμπεράσματα και σκέψεις για το μέλλον, που αποτελούν προτάσεις για μελλοντική επέκταση και βελτίωση των εφαρμογών.


Τα πρώτα Κεφάλαια \ref{c:complex} και \ref{c:finger} αποτελούν μια ανασκόπηση της θεωρίας των Κβαντικών Αποτυπωμάτων στο πλαίσιο της Πολυπλοκότητας της Επικοινωνίας. Το Κεφάλαιο \ref{c:complex} ορίζει βασικές έννοιες στην Κλασσική και Κβαντική Πολυπλοκότητα Επικοινωνίας, ενώ το Κεφάλαιο \ref{c:finger} πραγματεύεται ειδικότερα τα Κβαντικά Αποτυπώματα. Στο Κεφάλαιο \ref{c:crypto} διερευνούμε τα Κβαντικά Αποτυπώματα από τη σκοπιά της Κβαντικής Κρυπτογραφίας, εξηγούμε την ικανότητά τους στην απόκρυψη πληροφορίας και επιδεικνύουμε  τη χρήση τους ως Κβαντικές Ψηφιακές Υπογραφές. Προτείνουμε ένα σχήμα Κβαντικών Χρημάτων δημοσίου κλειδιού (public-key), το οποίο αποδεικνύεται ασφαλές έναντι κλασσικών και κβαντικών επιθέσεων. Τέλος, στο Κεφάλαιο \ref{c:exper} εξετάζουμε τις πειραματικές υλοποιήσεις των Κβαντικών Αποτυπωμάτων που έχουν προταθεί στην επιστημονική κοινότητα εστιάζοντας ιδιαίτερα σε μια πρόσφατη πειραματική διαδικασία που χρησιμοποιεί σύμφωνες καταστάσεις φωτός (coherent states) στην κατάσταση λειτουργίας time-bin (time-bin mode) \cite{constant}, η οποία καθιστά το πρωτόκολλο Κβαντικών Χρημάτων που προτείνουμε, εν μέρει πειραματικώς υλοποιήσιμο.

Στο πρώτο κεφάλαιο γίνεται μία αναφορά στο πώς οι μεθοδολογίες που αναπτύχθηκαν στα πλαίσια της διπλωματικής εργασίας μπορούν να χρησιμοποιηθούν σε πραγματικές εφαρμογές επαυξημένης πραγματικότητας.

