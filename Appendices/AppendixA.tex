

\chap{Γραμμική Άλγεβρα}

Τα βασικά αντικείμενα τα οποία πραγματεύεται η Γραμμική Άλγεβρα και αφορούν την Κβαντομηχανική είναι οι διανυσματικοί χώροι με εσωτερικό γινόμενο (Hilbert spaces). Ένας τέτοιος χώρος είναι για παράδειγμα ο $\mathbb{C}^n$ , ο χώρος δηλαδή όλων των n-άδων $(z_1,\ldots,z_n) z \in \mathbb{C}$. Τα στοιχεία του χώρου ονομάζονται διανύσματα και αρκετές φορες συμβολίζονται με τη μορφή ενός πίνακα μιας στήλης $1 \times n$. Στην κβαντομηχανική ωστόσο ακολουθείται ο συμβολισμός του DIrac , γνωστός και ως bra-ket notation.

Το σύμβολο $\ket{\cdot}$ (ket) χρησιμοποιείται για να δηλώσει ένα διάνυσμα, δηλαδή: \[\ket{\psi} \equiv  (y_1,\ldots,y_n) \equiv  \begin{bmatrix}y_1\\\vdots\\y_n\end{bmatrix}\] Το μηδενικό διάνυσμα συμβολίζεται με $0$ ώστε να μην συγχύεται με την θεμελιώδης κατάσταση του κβαντικού bit η οποία συμβολίζεται $\ket{0}$ .

Το σύμβολο $\bra{\cdot}$ (bra) συμβολίζει τον μιγαδικό ανάστροφο του αντίστοιχου ket. Δηλαδή: \[\bra{\psi} \equiv  \begin{bmatrix}y_1^{*}&\ldots&y_n^{*}\end{bmatrix}\]

Γραμμικός τελεστής (μετασχηματισμός) μεταξύ δύο΄διανυσματικών χώρων $\mathbb{V},\mathbb{W}$ λέγεται κάθε συνάρτηση $A : \mathbb{V} \rightarrow \mathbb{W}$ που είναι γραμμικώς ως προς τις εισόδους.
Δηλαδή: $ A(\sum_i\alpha_i\ket{v_i} = \sum_i\alpha_iA(\ket{v_i}) $ και συμβολίζεται $A\ket{v_i}$.

Ένας γραμμικός τελεστής εναλλακτικά μπορεί να παρασταθεί με έναν πίνακα $m \times n$ ή ισοδύναμα ένας πίνακας $m \times n$ αποτελεί έναν γραμ. τελεστή (μετασχηματισμό) της μορφής $A : \mathbb{C}^n \rightarrow \mathbb{C}^m$. Τέσσερις τελεστές ιδιαίτερης σημασίας για την κβαντομηχανική είναι οι τελεστές του Pauli ,όπως παρουσιάζονται παρακάτω:
\[
\sigma_0 \equiv  \mathbb{I} \equiv 
\begin{bmatrix}
1&0\\
0&1\\
\end{bmatrix}  
\ \ \ \sigma_1 \equiv  \sigma_x \equiv 
\begin{bmatrix}
0&1\\
1&0\\
\end{bmatrix}\]\[
\sigma_2 \equiv  \sigma_y \equiv 
\begin{bmatrix}
0&-i\\
i&0\\
\end{bmatrix}  
\ \ \ \sigma_3 \equiv  \sigma_z \equiv 
\begin{bmatrix}
1&0\\
0&-1\\
\end{bmatrix}
\]

